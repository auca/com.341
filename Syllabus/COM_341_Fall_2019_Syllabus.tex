\documentclass[12pt,a4paper,oneside]{article}

\usepackage[margin=3cm]{geometry}

\usepackage{hyperref}
\hypersetup{
    pdftitle={COM 341, Operating Systems},%
    pdfauthor={Toksaitov Dmitrii Alexandrovich},%
    pdfsubject={Syllabus},%
    pdfkeywords={COM;}{341;}{syllabus;}{operating;}{systems},%
    colorlinks,%
    linkcolor=black,%
    citecolor=black,%
    filecolor=black,%
    urlcolor=black
}

\newcommand{\R}[1]{\uppercase\expandafter{\romannumeral #1\relax}}

\begin{document}

    \title{COM 341, Operating Systems}
    \author{
        American University of Central Asia\\
        Software Engineering Department
    }
    \date{}
    \maketitle

    \section{Course Information}

        \begin{description}
            \item[Course Code]\hfill\\
                COM 341
            \item[Course ID]\hfill\\
                3325
            \item[Prerequisite]\hfill\\
                None
            \item[Credits]\hfill\\
                6
            \item[Professors, TAs, Time, Place]\hfill\\
                Lecture (Dmitrii Toksaitov): Monday 10:50--12:05, 220\\
                Lab (Dmitrii Toksaitov): Monday 12:45--14:00, G31\\
                Lab (Dmitrii Toksaitov): Monday 14:10--15:25, G31\\
                Lab (Dmitrii Toksaitov): Wednesday 14:10--15:25, G31\\
                TA Consultations (Bektur Umarbaev): By appointment
            \item[Course Repository]\hfill\\
                \url{https://github.com/auca/com.341}
            \item[Class Discussions]\hfill\\
                \url{https://piazza.com/auca.kg/fall2019/com341}
        \end{description}

    \section{Contact Information}

        \begin{description}
            \item[Instructor]\hfill\\
                Toksaitov Dmitrii Alexandrovich\\
                \href{mailto:toksaitov_d@auca.kg}{toksaitov\_d@auca.kg}
            \item[Teacher Assistant]\hfill\\
                Bektur Umarbaev\\
                \href{mailto:umarbaev_b@auca.kg}{umarbaev\_b@auca.kg}
            \item[Office]\hfill\\
                AUCA, room 315
            \item[Office Hours]\hfill\\
                By appointment throughout the work week\\
                Remotely through Skype on Saturday and Sunday from 12:00 to
                18:00
        \end{description}

    \section{Course Overview}

        This course introduces students to the fundamentals of operating systems
        design and implementation. Topics include an overview of the components
        of an operating system, synchronization, implementation of processes,
        scheduling algorithms, memory management and file systems. Students will
        learn basics of the Unix environment, the C programming language, and
        the ARM/x86 assembly. These technologies will help them to finish lab
        and project tasks to build common Unix utilities, to study the concept
        of systems calls, to peak into the inner workings of the Linux kernel,
        and to implement a simple Fuse file system.

        As a result, students should be able to research and analyze the
        functioning of the information technology systems, improve their skills
        using programming languages for software design, development, and
        maintenance in accord to the goals of the AUCA Software Engineering
        Department and the 510300 IT competency standard (including competency
        elements OK 1–7, ИК 1–7, ПК 1–15).

    \section{Topics Covered}

        \begin{itemize}
            \item Week 1--2: Introduction, History, OS Concepts Overview, Terminals (6 hours)
            \item Week 3--5: System Calls (9 hours)
            \item Week 6: Protection Mechanisms (3 hours)
            \item Week 7--8: Interprocess Communication (6 hours)
            \item Week 9--10: Scheduling (6 hours)
            \item Week 11--12: Virtual Memory Management (6 hours)
            \item Week 13: Swapping (3 hours)
            \item Week 14--15: File System Implementation (6 hours)
            \item Week 16: RAM Disks, Disks (3 hours)
        \end{itemize}

    \subsection{Lectures}

        Students will have to take midterm and final examinations on topics
        discussed during lectures. Each examination is in the form of a quiz
        with a set of open and multiple choice questions.

    \section{Practice Tasks and Labs}

        Students are required to finish 2 practice tasks during the course.
        These tasks are based on topics discussed during lectures.

        Students will have to finish 10 lab tasks. In every task students will
        study a common Unix utility and try to implement it on their own.

    \section{Course Project}

        The course project is to develop utilities to service a toy file system
        to check and defragment it to improve performance.

    \section{Course Materials, Recordings and Screencasts}

        Students will find all the course materials on GitHub. We hope by
        working with GitHub students will become familiar with the Git version
        control system and the popular (among developers) GitHub service. Though
        version control is not the focus of the course, some course tasks may
        have to be submitted through it on the GitHub Classroom service.

        Every class is screencasted online and recorded to YouTube for
        students’ convenience. An ability to watch a class remotely MUST NOT be
        a reason to not attend the class. Active class participation is
        necessary to succeed on this course.

    \section{Reading}

        \begin{enumerate}
            \item Operating System Concepts, 10th Edition by Abraham
            Silberschatz (ISBN-13: 978-1119456339, ISBN-10: 1119456339)
        \end{enumerate}

        \subsection{Supplemental Reading}

            \begin{enumerate}
                \item Understanding the Linux kernel, Third Edition by Daniel P.
                Bovet and Marco Cesati (AUCA Library Call Number: QA76.76.O63
                B683 2006, ISBN: 978-0596005658)
                \item Linux Kernel Development, 3rd Edition by Robert Love
                (ISBN: 978-0672329463)
                \item Windows Internals, Part 1 (6th Edition) by Mark E.
                Russinovich and David A. Solomon (AUCA Library Call Number:
                QA76.76.W56 R885 2012, ISBN: 978-0735648739)
                \item Windows Internals, Part 2 (6th Edition) by Mark E.
                Russinovich and David A. Solomon (AUCA Library Call Number:
                QA76.76.W56 R885 2012, ISBN: 978-0735665873)
                \item Mac OS X and iOS internals : to the apple's core by
                Jonathan Levin (AUCA Library Call Number: QA76.774.M33 L48 2013,
                ISBN: 978-1118057650)
                \item Mac OS X Internals: A Systems Approach by Amit Singh (AUCA
                Library Call Number: QA76.76.O63 S564 2007, ISBN:
                978-0321278548)
            \end{enumerate}

    \section{Grading}

        \begin{itemize}
            \item Lecture Midterm (15\%)
            \item Lecture Final (15\%)
            \item Practice Midterm (17.5\%)
            \item Practice Final (22.5\%)
            \item Course Project (25\%)
            \item Piazza Participation (5\%)
        \end{itemize}

    \section{Scale}

        \begin{itemize} \itemsep-10pt \parskip0pt \parsep0pt
            \item[--] 92\%--100\%: A\\
            \item[--] 85\%--91\%: A-\\
            \item[--] 80\%--84\%: B+\\
            \item[--] 75\%--79\%: B\\
            \item[--] 70\%--74\%: B-\\
            \item[--] 65\%--69\%: C+\\
            \item[--] 60\%--64\%: C\\
            \item[--] 55\%--59\%: C-\\
            \item[--] 50\%--54\%: D+\\
            \item[--] 45\%--49\%: D\\
            \item[--] 40\%--44\%: D-\\
            \item[--] Less than 40\%: F
        \end{itemize}

    \section{Rules}

        Students are required to follow the rules of conduct of the Software
        Engineering Department and American University of Central Asia.

    \subsection{Participation}

        Active work during the class may be awarded with up to 5 extra points at
        the instructor’s discretion.

        Poor student performance during a class can lead to up to 5 points being
        deducted from the final grade.

        Instructors may conduct pop-checks during classes at random without
        prior notice. Students MUST be ready for every class in order not to
        loose points.

    \subsection{Attendance}

        Missing more than three classes without a reason will result in 10
        points being deducted from the student for every day. If a student has
        health/family/personal emergency, he MUST notify the instructor in
        advance (e.g., through e-mail). The student MUST also provide a valid
        proof afterwards. Without a prior notice and a valid proof the miss will
        still be counted.

    \subsection{Questions}

        We believe that a question from one student is most likely a question
        that other students are also interested in. That is why we encourage
        students to use Piazza to ask questions in public that other students
        can see and answer and NOT ask them through E-mail in private UNLESS the
        question itself is about private matters to discuss with the professor.

    \subsection{Late Policy}

        Late submissions and late exams are not allowed. Exceptions may be made
        at a discretion of the professor only in force-majeure circumstances.

    \subsection{Incomplete}

        As with late exams, the grade \textit{I} may be awarded only in special
        circumstances. The student must start discussion on getting the grade I
        with the instructors in advance and not during the last week before the
        final exams.

    \subsection{Academic Honesty}

        Plagiarism can be defined as “an act or an example of copying or
        stealing someone else’s words or ideas and appropriating them as one’s
        own”. The concept of plagiarism applies to all tasks and their
        components, including program code, abstracts, reports, graphs,
        statistical tables, etc.

        In addition to being unethical, this indicates that the student has not
        studied the given material. Tasks written from somewhere for 10\% or
        less will be assessed accordingly or will receive a 0 at the discretion
        of the teacher. If plagiarism is more than 10\%, the case will be
        transferred to the AUCA Disciplinary Committee.

        Students are not recommended to memorize before exams, as this is a
        difficult and inefficient way to learn; and since practice exams consist
        of open questions designed to test a student’s analytical skills,
        memorization invariably leads to the fact that the answers are
        inappropriate and of poor quality.

        On this course team work is NOT encouraged. The same blocks of code or
        similar structural pieces in separate submissions will be considered as
        academic dishonesty and all parties will get zero for the task.

        The following are examples of some common acts of plagiarism:

        \begin{enumerate}
            \item Representing the work of others as their own
            \item Using other people's ideas or phrases without specifying the
                  author
            \item Copying code snippets, sentences, phrases, paragraphs or ideas
                  from other people's works,published or unpublished, without
                  referring to the author
            \item Replacing selected words from a passage and using them as your
                  own
            \item Copying from any type of multimedia (graphics, audio, video,
                  Internet streams), computer programs, graphs or diagrams from
                  other people's works without representation of authorship
            \item Buying work from a website or from another source and
                  presenting it as your own work
        \end{enumerate}

\end{document}
